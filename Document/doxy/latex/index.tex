\begin{DoxyAuthor}{Authors}
Texas Instruments Software Developers
\end{DoxyAuthor}
\hypertarget{index_intro}{}\section{Introduction}\label{index_intro}
This package contains library software, including code templates for use by T\-P\-S23861 software developers.

The documentation is maintained as doxygen based comments throughout the source code.

The software is broken up into separate functions. There are public functions that are used by standard software developers to command and control the T\-P\-S23861.

Additionally, there are a set of abstraction level functions that are collected in the T\-P\-S23861\-\_\-\-Glue.\-c file. The T\-P\-S23871 resides on the I2\-C bus, and standard I2\-C commands are used to communicate with it. As different hardware processors and operating systems use different functions to perform I2\-C communications, the software here uses standard function calls which can be changed by the end user into the specific function calls for their specific applciation. This small set of hardware abstraction functions are all collected into one file for ease of porting.

For a number of functions there are two different functions that may perform the same configuration. One function exists to set up all four ports on the device at the same time. A seaparate function exists to configure one of the four ports. An application that has fixed connections may wish to configure everything at once. An application that sees unknown devices inserted and removed may wish to configure each port individually as they are discovered. In general, the funciotns that deal with individual ports will have the word Port in the function name. For example, tps\-\_\-\-Set\-Detect\-Class\-Enable() enables the detection and classification for all four ports. However, the \hyperlink{_t_p_s23861_8c_a60e64452aad48caf62d2e2e590471666}{tps\-\_\-\-Set\-Port\-Detect\-Class\-Enable()} function only enables one port's detection and classification capaibilities\hypertarget{index_license}{}\section{Document License}\label{index_license}
This work is licensed under the Creative Commons Attribution-\/\-Share Alike 3.\-0 United States License. To view a copy of this license, visit \href{http://creativecommons.org/licenses/by-sa/3.0/us/}{\tt http\-://creativecommons.\-org/licenses/by-\/sa/3.\-0/us/} or send a letter to Creative Commons, 171 Second Street, Suite 300, San Francisco, California, 94105, U\-S\-A.

Copyright (C) 2013 Texas Instruments Incorporated -\/ \href{http://www.ti.com/}{\tt http\-://www.\-ti.\-com/} 